\documentclass[12pt, a4paper]{article}
\usepackage{graphicx} % Required for inserting images
\usepackage{bm}
\usepackage{amsmath}
\usepackage[top=2cm, bottom=2cm, left=2cm, right=2cm]{geometry}
\usepackage{indentfirst}

\title{Localization EKF}
\author{Yi-Chen Zhang}
\date{\today}

\begin{document}

\maketitle

\section{State representation $\bm{x}$ and motion input $\bm{u}$}
Let $x$ and $y$ represent positions in the $x$- and $y$-directions, respectively. The variables $\theta$, $\nu$, $\omega$, and $\alpha$ denote the yaw angle, linear velocity, yaw angle rate, and linear acceleration, respectively. The target state $\bm{x}$ is defined as:
\[
  \bm{x} = [x, y, \theta, \nu, \omega, \alpha]^{T}
\]
The motion input is defined as:
\[
  \bm{u} = [\omega, \alpha]^{T}
\]

\section{The measurement $\bm{z}$}
Consider a scenario where we have both a GPS sensor and an IMU sensor. The GPS sensor provides information about the longitudinal position ($x$) and lateral position ($y$). On the other hand, the IMU sensor supplies data regarding the yaw angle ($\theta$), yaw angle rate ($\omega$), and linear acceleration ($\alpha$). Additionally, the vehicle itself provides its velocity ($\nu$). We can define the measurement vector $\bm{z}$ as follows:
%Suppose we have a GPS sensor and IMU sensor. The GPS sensor provides the longitudinal position $x$ and lateral position $y$. The IMU sensor provides yaw angle $\theta$, yaw angle rate $\omega$, and linear acceleration $\alpha$. The vehicle also provides it's velocity $\nu$. The measurement $\bm{z}$ is defined as:
\[
  \bm{z} = [x, y, \theta, \omega, \alpha, \nu]^{T}
\]
This representation captures the key measurements of interest.

\section{State transition function $\bm{f}$ and its derivative}
The 2-D kinematic equation for the vehicle state at time $t$, denoted as $\bm{x}_t$ and described by kinematic model in discrete time space, can be expressed as follows:
\begin{equation}
  \label{eq:state_transition}
  \begin{bmatrix}
    x_{t}\\
    y_{t}\\
    \theta_{t}\\
    \nu_{t}\\
    \omega_{t}\\
    \alpha_{t}
  \end{bmatrix}=
  \begin{bmatrix}
    x_{t-1} + (\nu_{t-1}\Delta t + \frac{1}{2}\alpha_{t-1}\Delta t^2) \cos(\theta_{t-1}+\frac{1}{2}\omega_{t-1}\Delta t) + \epsilon_{x}\\
    y_{t-1} + (\nu_{t-1}\Delta t + \frac{1}{2}\alpha_{t-1}\Delta t^2) \sin(\theta_{t-1}+\frac{1}{2}\omega_{t-1}\Delta t) + \epsilon_{y}\\
    \theta_{t-1} + \omega_{t-1}\Delta t + \epsilon_{\theta}\\
    \nu_{t-1} + \alpha_{t-1} \Delta t + \epsilon_{\nu}\\
    \omega_{t-1}+\epsilon_{\omega}\\
    \alpha_{t-1}+\epsilon_{\alpha}
  \end{bmatrix}
\end{equation}
The state transition equation \eqref{eq:state_transition} can be expressed through the nonlinear function $\bm{f}$ as follows:
%The above equation \eqref{eq:state_transition} can be expressed as the non-linear function $\bm{f}$:
\[
  \bm{x}_{t} = \bm{f}(\bm{x}_{t-1}, \bm{u}_{t},\bm{\epsilon}_{t})
\]
Here, the vector $\bm{\epsilon}_{t}=[\epsilon_{x}, \epsilon_{y}, \epsilon_{\theta}, \epsilon_{\nu}, \epsilon_{\omega}, \epsilon_{\alpha}]^{T}$ represents the system noise. To simplify our analysis, we assume that system noises $\epsilon_{x}$, $\epsilon_{y}$, $\epsilon_{\theta}$, and $\epsilon_{\nu}$ are set to zero, making $\bm{\epsilon}_{t}$ solely associated with the motion input $\bm{u}_{t}$. Furthermore, the nonlinear function $\bm{f}$ can be approximated through a Taylor expansion around the previously updated state $\bm{\mu}_{t-1}$, with $\bm{\epsilon}_{t}=\bm{0}$:

%Here, the vector $\bm{\epsilon}_{t}=[\epsilon_{x}, \epsilon_{y}, \epsilon_{\theta}, \epsilon_{\nu}, \epsilon_{\omega}, \epsilon_{\alpha}]^{T}$ represents the system noise. Since the system noises are originated from the motion input $\bm{\mu}_{t}$, we make further assumption that $\epsilon_{x}$, $\epsilon_{y}$, $\epsilon_{\theta}$, and $\epsilon_{\nu}$ are all set to zero, ensuring that $\bm{\epsilon}_{t}$ is solely associated with the motion input $\bm{u}_{t}$. Additionally, the non-linear function $\bm{f}$ can be approximated using a Taylor expansion at the previously updated state $\bm{\mu}_{t-1}$, with $\bm{\epsilon}_{t}=\bm{0}$:
\[
  \bm{f}(\bm{x}_{t-1}, \bm{u}_{t}, \bm{\epsilon}_{t}) \approx \bm{f}(\bm{\mu}_{t-1}, \bm{u}_{t}, \bm{0})+\bm{F}_{t}(\bm{x}_{t-1}-\bm{\mu}_{t-1})+\bm{W}_{t}\bm{\epsilon}_{t}
\]
Here, $\bm{F}_{t}$ represents the derivative of the state transition function $\bm{f}$ with respect to state $\bm{x}$, and $\bm{W}_{t}$ represents the derivative of the state transition function $\bm{f}$ with respect to the noise $\bm{\epsilon}$. The term $\bm{W}_{t}$ essentially denotes the noise gain of the system. To simplify the notation, we define $\Delta l$ as $\nu_{t-1}\Delta t + \frac{1}{2}\alpha_{t-1}\Delta t^2$ and $\Delta\theta$ as $\theta_{t-1} + \frac{1}{2}\omega_{t-1}\Delta t$. The derivative matrix of the state transition function $\bm{F}_{t}$ is denoted as:
\begin{eqnarray*}
  \bm{F}_{t} & = & \frac{\partial\bm{f}}{\partial\bm{x}}\\
  %& = & \begin{bmatrix}
    %\frac{\partial f_{x}}{\partial x} & \frac{\partial f_{x}}{\partial y} & \frac{\partial f_{x}}{\partial \theta} & \frac{\partial f_{x}}{\partial \nu} & \frac{\partial f_{x}}{\partial \omega} & \frac{\partial f_{x}}{\partial \alpha}\\
    %\frac{\partial f_{y}}{\partial x} & \frac{\partial f_{y}}{\partial y} & \frac{\partial f_{y}}{\partial \theta} & \frac{\partial f_{y}}{\partial \nu} & \frac{\partial f_{y}}{\partial \omega} & \frac{\partial f_{y}}{\partial \alpha}\\
    %\frac{\partial f_{\theta}}{\partial x} & \frac{\partial f_{\theta}}{\partial y} & \frac{\partial f_{\theta}}{\partial \theta} & \frac{\partial f_{\theta}}{\partial \nu} & \frac{\partial f_{\theta}}{\partial \omega} & \frac{\partial f_{\theta}}{\partial \alpha}\\
    %\frac{\partial f_{\nu}}{\partial x} & \frac{\partial f_{\nu}}{\partial y} & \frac{\partial f_{\nu}}{\partial \theta} & \frac{\partial f_{\nu}}{\partial \nu} & \frac{\partial f_{\nu}}{\partial \omega} & \frac{\partial f_{\nu}}{\partial \alpha}\\
    %\frac{\partial f_{\omega}}{\partial x} & \frac{\partial f_{\omega}}{\partial y} & \frac{\partial f_{\omega}}{\partial \theta} & \frac{\partial f_{\omega}}{\partial \nu} & \frac{\partial f_{\omega}}{\partial \omega} & \frac{\partial f_{\omega}}{\partial \alpha}\\
    %\frac{\partial f_{\alpha}}{\partial x} & \frac{\partial f_{\alpha}}{\partial y} & \frac{\partial f_{\alpha}}{\partial \theta} & \frac{\partial f_{\alpha}}{\partial \nu} & \frac{\partial f_{\alpha}}{\partial \omega} & \frac{\partial f_{\alpha}}{\partial \alpha}
  %\end{bmatrix}\\
  & = & \begin{bmatrix}
  1 & 0 & -\Delta l\sin(\Delta\theta) & \Delta t\cos(\Delta\theta) & -\frac{1}{2}\Delta t\Delta l\sin(\Delta\theta) & \frac{1}{2}\Delta t^2\cos(\Delta\theta)\\
  0 & 1 & \Delta l\cos(\Delta\theta) & \Delta t\sin(\Delta\theta) & \frac{1}{2}\Delta t\Delta l\cos(\Delta\theta) & \frac{1}{2}\Delta t^2\sin(\Delta\theta)\\
  0 & 0 & 1 & 0 & \Delta t & 0\\
  0 & 0 & 0 & 1 & 0 & \Delta t\\
  0 & 0 & 0 & 0 & 1 & 0\\
  0 & 0 & 0 & 0 & 0 & 1
  \end{bmatrix}
\end{eqnarray*}
Furthermore, $\bm{W}_{t}$ can be expressed in matrix form:
\begin{eqnarray*}
  \bm{W}_{t} & = & \frac{\partial\bm{f}}{\partial\bm{\epsilon}}\\
  %& = & \begin{bmatrix}
  %  \frac{\partial f_{x}}{\partial \nu} & \frac{\partial f_{x}}{\partial \omega} & \frac{\partial f_{x}}{\partial \alpha}\\
  %  \frac{\partial f_{y}}{\partial \nu} & \frac{\partial f_{y}}{\partial \omega} & \frac{\partial f_{y}}{\partial \alpha}\\
  %  \frac{\partial f_{\theta}}{\partial \nu} & \frac{\partial f_{\theta}}{\partial \omega} & \frac{\partial f_{\theta}}{\partial \alpha}\\
  %  \frac{\partial f_{\nu}}{\partial \nu} & \frac{\partial f_{\nu}}{\partial \omega} & \frac{\partial f_{\nu}}{\partial \alpha}\\
  %  \frac{\partial f_{\omega}}{\partial \nu} & \frac{\partial f_{\omega}}{\partial \omega} & \frac{\partial f_{\omega}}{\partial \alpha}\\
  %  \frac{\partial f_{\alpha}}{\partial \nu} & \frac{\partial f_{\alpha}}{\partial \omega} & \frac{\partial f_{\alpha}}{\partial \alpha}
  %\end{bmatrix}\\
  & = & \begin{bmatrix}
    0 & 0 & 0 & 0 & -\frac{1}{2}\Delta t\Delta l\sin(\Delta\theta) & \frac{1}{2}\Delta t^{2}\cos(\Delta\theta)\\
    0 & 0 & 0 & 0 & \frac{1}{2}\Delta t\Delta l\cos(\Delta\theta) & \frac{1}{2}\Delta t^{2}\sin(\Delta\theta)\\
    0 & 0 & 0 & 0 & \Delta t & 0\\
    0 & 0 & 0 & 0 & 0 & \Delta t\\
    0 & 0 & 0 & 0 & 1 & 0\\
    0 & 0 & 0 & 0 & 0 & 1
  \end{bmatrix}
\end{eqnarray*}
%We will derive the process noise covariance $\bm{Q}_{t}$ in section \ref{sec:process_noise}.

\section{Measurement function $\bm{h}$ and its derivative}
The measurements of the system can be described by a linear relation between the state and the measurement at the current step $t$. This relation is expressed as:
\begin{equation}
  \label{eq:measurement_model}
  \begin{bmatrix}
    z_{x}\\
    z_{y}\\
    z_{\theta}\\
    z_{\omega}\\
    z_{\alpha}\\
    z_{\nu}
  \end{bmatrix}=
  \begin{bmatrix}
    x_{t} + \delta_{x}\\
    y_{t} + \delta_{y}\\
    \theta_{t} + \delta_{\theta}\\
    \omega_{t} + \delta_{\omega}\\
    \alpha_{t} + \delta_{\alpha}\\
    \nu_{t} + \delta_{\nu}
  \end{bmatrix}
\end{equation}
We assume that the measurements are associated with measurement noises, and the noises are independent of each other. The above equation \eqref{eq:measurement_model} can be expressed as the linear function $\bm{h}$:

\[
  \bm{z}_{t} = \bm{h}(\bm{x}_{t}, \bm{\delta}_{t})
\]

Here, $\bm{\delta}_{t}=[\delta_{x}, \delta_{y}, \delta_{\theta}, \delta_{\omega}, \delta_{\alpha}, \delta_{\nu}]^{T}$ represents the measurement error. We can apply the same technique to expend the function $\bm{h}$ at the previously predicted state $\bar{\bm{\mu}}_{t}$, with $\bm{\delta}_{t}=\bm{0}$:
\[
  \bm{h}(\bm{x}_{t},\bm{\delta}_{t}) \approx \bm{h}(\bar{\bm{\mu}}_{t}, \bm{0}) + \bm{H}_{t}(\bm{x}_{t}-\bar{\bm{\mu}}_{t}) + \bm{V}_{t}\bm{\delta}_{t}
\]
Here, $\bm{H}_{t}$ represents the derivative of the measurement function $\bm{h}$ with respect to $\bm{x}$, and $\bm{V}_{t}$ represents the derivative of the measurement function $\bm{h}$ with respect to the noise $\bm{\delta}$. The term $\bm{V}_{t}$ is the noise gain of the measurement. The derivative matrix of the measurement function is denoted as:
\begin{eqnarray*}
  \bm{H}_{t} & = & \frac{\partial\bm{h}}{\partial{\bm{x}}}\\ 
  & = & \begin{bmatrix}
    1 & 0 & 0 & 0 & 0 & 0\\
    0 & 1 & 0 & 0 & 0 & 0\\
    0 & 0 & 1 & 0 & 0 & 0\\
    0 & 0 & 0 & 0 & 1 & 0\\
    0 & 0 & 0 & 0 & 0 & 1\\
    0 & 0 & 0 & 1 & 0 & 0\\
  \end{bmatrix}.
\end{eqnarray*}
In addition, $\bm{V}_{t}$ can be expressed in matrix form:
\begin{eqnarray*}
  \bm{V}_{t} & = & \frac{\partial \bm{h}}{\partial \bm{\delta}}\\
  & = & \begin{bmatrix}
    1 & 0 & 0 & 0 & 0 & 0\\
    0 & 1 & 0 & 0 & 0 & 0\\
    0 & 0 & 1 & 0 & 0 & 0\\
    0 & 0 & 0 & 1 & 0 & 0\\
    0 & 0 & 0 & 0 & 1 & 0\\
    0 & 0 & 0 & 0 & 0 & 1
  \end{bmatrix}
\end{eqnarray*}

\section{Process noise covariance matrix $\bm{Q}$}
% Yi-Chen
As we mentioned earlier, we assume that the system noise $\bm{\epsilon}_{t}$ is associated with the motion input $\bm{u}_{t}$. Furthermore, we assume that $\bm{\epsilon}_{t}$ follows a normal distribution with a mean of $\bm{0}$ and a covariance matrix of $\bm{Q}_{t}$. Additionally, we define the covariance matrix $\bm{Q}_{t}$ as
\[
  \bm{Q}_{t}=\begin{bmatrix}
    0 & 0 & 0 & 0 & 0 & 0\\
    0 & 0 & 0 & 0 & 0 & 0\\
    0 & 0 & 0 & 0 & 0 & 0\\
    0 & 0 & 0 & 0 & 0 & 0\\
    0 & 0 & 0 & 0 &\sigma_{\omega}^{2} & 0\\
    0 & 0 & 0 & 0 &0 & \sigma_{\alpha}^{2}
  \end{bmatrix}
\]
Here, $\sigma_{\omega}$ and $\sigma_{\alpha}$ are the standard deviation of the yaw angle rate and linear acceleration, respectively. %The process noise covariance matrix $\bm{Q}_{t}$, representing the error in the state process, can be expressed as follows:
% \[
%   \bm{Q}_{t} = \bm{W}_{t}\bm{\Omega}_{t}\bm{W}_{t}^{T}
% \]

\section{Measurement noise covariance matrix $\bm{R}$}
For the measurement error $\bm{\delta}_{t}$, we assume it follows a normal distribution with a mean of $\bm{0}$ and a covariance matrix of $\bm{R}_{t}$. Additionally, we assume that all measurement noises are independent. Consequently, we can disregard any interaction between them, resulting in the following diagonal covariance matrix:
\[
  \bm{R}_{t} = \begin{bmatrix}
    \tau_{x}^2 & 0 & 0 & 0 & 0 & 0\\
    0 & \tau_{y}^2 & 0 & 0 & 0 & 0\\
    0 & 0 &\tau_{\theta}^2 & 0 & 0 & 0\\
    0 & 0 & 0 & \tau_{\omega}^2 & 0 & 0\\
    0 & 0 & 0 & 0 & \tau_{\alpha}^2 & 0\\
    0 & 0 & 0 & 0 & 0 & \tau_{\nu}^2\\
  \end{bmatrix}.
\]
Here, $\tau_{x}$, $\tau_{y}$, $\tau_{\theta}$, $\tau_{\omega}$, $\tau_{\alpha}$, and $\tau_{\nu}$ represent the standard deviations of $x$, $y$, $\theta$, $\omega$, $\alpha$, and $\nu$ respectively.

\section{Kalman filter algorithm}
Now we are ready to implement extended Kalman filter (EKF) for the vehicle localization. Given the initial state $\bm{\mu}_{0}$ and state covariance $\bm{\Sigma}_{0}$, the EKF algorithm is summarized as follows:\\

\textbf{Prediction}:
\begin{center}
\begin{tabular}{ll}
  Predicted state estimate: & $\bar{\bm{\mu}}_{t} = \bm{f}_{t}(\bm{\mu}_{t-1}, \bm{u}_{t})$\\
  Predicted error covariance: & $\bar{\bm{\Sigma}}_{t} = \bm{F}_{t}\bm{\Sigma}_{t-1}\bm{F}_{t}^{T}+\bm{W}_{t}\bm{Q}_{t}\bm{W}_{t}^{T}$
\end{tabular}
\end{center}

\textbf{Update}:
\begin{center}
\begin{tabular}{ll}
  Innovation: & $\bm{y}_{t} = \bm{z}_{t}-\bm{h}(\bar{\bm{\mu}}_{t})$\\
  Innovation covariance: & $\bm{S}_{t} = \bm{H}_{t}\bar{\bm{\Sigma}}_{t}\bm{H}_{t}^{T}+\bm{V}_{t}\bm{R}_{t}\bm{V}_{t}^{T}$\\
  Kalman gain: & $\bm{K}_{t} = \bar{\bm{\Sigma}}_{t}\bm{H}_{t}^{T}\bm{S}_{t}^{-1}$\\
  Updated state estimate: & $\bm{\mu}_{t} = \bar{\bm{\mu}}_{t} + \bm{K}_{t}\bm{y}_{t}$\\
  Updated error covariance: & $\bm{\Sigma}_{t} = (\bm{I}-\bm{K}_{t}\bm{H}_{t})\bar{\bm{\Sigma}}_{t}$
\end{tabular}
\end{center}

For the updated error covariance, the Joseph formula should be employed for numerical stability. This can be expressed as follows:
\[
  \bm{\Sigma}_{t} = (\bm{I}-\bm{K}_{t}\bm{H}_{t})\bar{\bm{\Sigma}}_{t}(\bm{I}-\bm{K}_{t}\bm{H}_{t})^{T}+\bm{K}_{t}\bm{V}_{t}\bm{R}_{t}\bm{V}_{t}^{T}\bm{K}_{t}^{T}
\]

In the case where the measurement model is correct, the Kalman filter utilizes it for updates. Thus, a conditional statement for data association is introduced. The Mahalanobis distance is calculated for the measurement residual to determine if the measurement is suitable for updating:
\[
  (\bm{z}_{t}-\bm{h}(\bar{\bm{\mu}}_{t}))^{T}\bm{S}_{t}^{-1}(\bm{z}_{t}-\bm{h}(\bar{\bm{\mu}}_{t})) \leq D_{th},
\]
Here, $D_{th}$ represents a predetermined threshold. One can further show that the quadratic term is actually a $\chi_{m}^{2}$ distribution, where $m$ is the number of measurements.

\end{document}
